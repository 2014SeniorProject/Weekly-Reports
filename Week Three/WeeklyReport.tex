%| Weekly report template for CSUS Senior Design
%|
%| language: LaTeX
%| Author: Ben Smith
%| 
%| This source has been tagged with the "<CHANGE" tag in areas
%| that require updating when making a new docuent
%|
%| This source will generate a PDF file complete with thumbnails navigation menu and metadata.
%| Much of the tex awesomeness comes from http://www.michaelshell.org/ praise be to him for creating the guide

\documentclass[12pt,compsoc]{IEEEtran}

%| Override compsoc class' Palatino font for body text, restores to Times New Roman
\renewcommand{\rmdefault}{ptm}\selectfont

%| IEEE Citation package
 \usepackage{cite}

%| American Mathematical Society package for fancy maths	
\usepackage[cmex10]{amsmath}
\interdisplaylinepenalty=2500				% Restores IEEE line spacing after amsmath

%| Better tables than LaTeX 2e
\usepackage{array}

%| Improved URL handling
\usepackage{url}

%| Enables PDF metadata, thumbnails, and navigation
\newcommand\MYhyperrefoptions{
	bookmarks=true,
	bookmarksnumbered=true,
	pdfpagemode={UseOutlines},
	plainpages=false,
	pdfpagelabels=true,
	colorlinks=true,
	linkcolor={black},
	citecolor={black},
	pagecolor={black},
	urlcolor={black},
	pdftitle={Senior Design: Weekly Progress Report},							%#CHANGE
	pdfsubject={Senior Design},													%#CHANGE
	pdfauthor={Michael Frith, David Larribas, Devin Moore, Benjamin Smith},		%#CHANGE
	pdfkeywords={Weekly Report}}												%#CHANGE

%| Calls hyper ref package 
\usepackage[\MYhyperrefoptions,pdftex]{hyperref}

% correct bad hyphenation here
\hyphenation{op-tical net-works semi-conduc-tor}

\begin{document}

%| =================================================================================================
%| Title Block
%| =================================================================================================
\title{Senior Design: Week Three Progress Report}
\author{Team 3: Micheal Frith, David Larribas, Devin Moore, Benjamin Smith}
\date{\today}
\maketitle

%| =================================================================================================
%| Header 
%| =================================================================================================
\markboth{Progress Report 1: \today}%
{derp}

%| =================================================================================================
%| Weekly time report
%| =================================================================================================
\section{Group Time Synopsis}
	\subsection{Team efforts and overall status}
		{\bfseries Current Group Leader:} Ben Smith
		\IEEEPARstart{T}{he} group spent about half of its bandwidth on developing the work breakdown 
		structure (WBS) and documentation. The WBS gave the group an opportunity to clarify a feature set 
		after last week’s Design Document review. The Design Document revisions for this week reflect our 
		changes of expectation in accordance with last week's review.
		
		Three development boards have been purchased by the group for Verilog development and testing. 
		These boards come equipped with the Analog Devices ADXL345 accelerometer that is configurable for 
		SPI or I$^2$C. This accelerometer is extremely common and the most likely to be used in the final 
		production design. Having the accelerometer on board will prove invaluable for the breadboard proof.

		The development board supports JTAG over USB and does not require an external power supply. It will 
		provide hardware based verification in a small, easy to use form factor. The board is also based on 
		the Altera Cyclone IV FPGA. This allows the use of development tools familiar to the group members. 
		The FPGA is also large enough to use Signaltap, Altera’s embedded logic analyzer.
	
	\begin{table}[ht]
		\renewcommand{\arraystretch}{1.3}
		\caption{Weekly Team Meetings}
		
		\label{Team Hour Summary}
		
		\centering
		\begin{tabular}{p{7cm}|p{1cm}}
		\hline
		{\bfseries 	Topic} 																		&{\bfseries Hours}		\\
		\hline\hline
					{\bfseries Tuesday:}  Review after Meeting with Professor Tatro  			& 1.0					\\
					{\bfseries Thursday:} WBS assignment review and section assignment			& 3.0					\\
					{\bfseries Sunday :}  Finalization of WBS and Design Document Review		& 5.0					\\
					{\bfseries Monday :}  Weekly Report drafting								& 5.0					\\
					{\bfseries Monday :}  WBS typesetting and final review 						& 5.0					\\
		\hline
		\end{tabular}
	\end{table}
		 
	\subsection{Review after Meeting with Professor Tatro}
		We revised the punch list, particularly the motor control modes, after design document review.
		This lead to the elimination of all operational modes except biometric feedback. The biometric
		feedback scheme's definition was made more broad to allow for change over span of the project.
				
	\subsection{WBS assignment review and section assignment}
		Group members took responsibility for individual sections in the WBS related to their technical
		expertise. A diagrammatic representation of the WBS was constructed to ensure mutual exclusivity
		of the individual sections.

	\subsection{Finalization of WBS and Design Document Review}
		The group met on line to compile and review their individual sections of the WBS they worked on
		over the week. Final draft was prepared for typesetting on Monday.

	\subsection{Weekly Report drafting and weekly debrief}
		Group assembled to review progress over the last week. Expectations of next week's time expenditure
		were drafted from the lecture notes for the week. The group was brought up to speed with Mike's
		motor research from last week.

	\subsection{WBS typesetting and final review}
		Document was typeset and reviewed by the group to catch spelling and grammatical errors.

%| =================================================================================================
%| Individual Activity Synopsis
%| Mike
%| =================================================================================================
\section{Individual Activity Synopsis}
	\subsection{Micheal Frith}

	\subsubsection*{Synopsis of past week's work}
        I spent considerable time researching various hub motors as well as preliminarily selecting a 
        speed controller after getting in touch with Richard Lyen, a custom motor controller designer 
        in San Francisco. He steered me towards some potential motor suppliers as well as made some 
        recommendations for motors themselves. I'm waiting on contact back after some initial 
        conversation with a potential supplier about getting a discounted motor. Other than motor and 
        speed controller research, I spent time investigating integrating audio into our design. 
        Generating sound at the FPGA level may be a good preliminary implementation. Additionally 
        I worked on my sections of WBS document, weekly report, and design contract.

    \begin{table}[ht]
		\renewcommand{\arraystretch}{1.3}
		\caption{Micheal Frith: Tasks Assigned - Last Week}
		
		\label{Summary of Micheal Frith's activities: last week}
		
		\centering
		\begin{tabular}{p{5.5cm}|p{1cm}|p{1cm}}
		\hline
		\bfseries 	Task		 							& \bfseries Hours Worked	& \bfseries Status	\\
		\hline\hline
					Component Research/Selection			& 6.0 						& 60\%	 			\\    		
					Work Breakdown Structure                & 6.0 		                & 100\%   			\\   
					Researched implementing FPGA audio 		& 2.0						& 20\% 				\\
					Design Contract Finalization			& 4.0 						& 100\%   			\\  
					Gantt Chart								& 0 						& 10\%    			\\ 
							\hline
		\end{tabular}
	\end{table}

	\begin{table}[ht]
	\renewcommand{\arraystretch}{1.3}
		\caption{Micheal Frith: Tasks Assigned - This Week}
		
		\label{Summary of Micheal Frith's activities: this week}
		
		\centering
		\begin{tabular}{p{5.5cm}|p{1cm}|p{1cm}}
		\hline
		\bfseries 	Task		 							& \bfseries Hours Worked	& \bfseries Status	\\
		\hline\hline
					Component research/selection			& 6.0						& 80\%				\\
					PERT/TeamGantt							& 6.0						& 90\%				\\
                    Start FPGA Dev: Screen, User inputs     & 3                         & 10\%%             \\
		\hline
		\end{tabular}
	\end{table}

	%\subsubsection*{Special Problems or other reporting}
	%Getting a process established for getting work done more efficiently has been challenging, as we've each been spending 30+ hours a week these 2 weeks.

%| =================================================================================================
%| Individual Activity Synopsis
%| David
%| =================================================================================================
\subsection{David Larribas}

	\subsubsection*{Synopsis of past week's work}

	For the first task of this week, my group and I met up with Professor Tatro to realign our focus. 
	This led to some rewording that altered the scope of our project including changing specifications 
	in my part of the design document. I researched over a dozen lighting systems and their reviews to 
	estimate which LEDs to use. I researched three power conversion methods and settled on a buck 
	converter to be the ideal method because of its minimal power loss. I then wrote up the safety and 
	power system sections of the WBS diagram. I also wrote my sections of the WBS document including 
	the lighting and power systems. I had to research the various components of these sections by 
	searching through half a dozen websites to price these items.

	\begin{table}[ht]
	\renewcommand{\arraystretch}{1.3}
		\caption{David Larribas: Tasks Assigned - Last Week}
		
		\label{Summary of David Larribas' activities: last week}
		
		\centering
		\begin{tabular}{p{5.5cm}|p{1cm}|p{1cm}}
		\hline
		\bfseries 	Task		 	                        				& \bfseries Hours Worked	& \bfseries Status	\\
		\hline\hline
					Work breakdown Structure 								& 8.0				    	& 100\%				\\	
					Develop Gantt Chart										& 1.0						& 5\%               \\
                    Component research and selection for lighting 			& 3.0                       & 50\%              \\
                    Component research and selection for Voltage Regulator	& 2.0                       & 25\%              \\
                    Weekly Report - Week 3                                   & 2.0                       & 100\%         \\
                    Editing Design Document                                 & 2.0                       & 100\%
		\hline
		\end{tabular}
	\end{table}

	\begin{table}[ht]
	\renewcommand{\arraystretch}{1.3}
		\caption{David Larribas: Tasks Assigned - Next Week}
		
		\label{Summary of David Larribas' activites: this week}
		
		\centering
		\begin{tabular}{p{5.5cm}|p{1cm}|p{1cm}}
		\hline

		\bfseries 	Task		 	                         			    & \bfseries Hours Worked	& \bfseries Status	\\
		\hline\hline
					Develop Gantt Chart										& 4.0						& 5\%				\\
                    Component research and selection for Lighting           & 2.0                       & 50\%              \\
					Component research and selection for Voltage Regulator  & 6.0						& 25\%				\\
					"Breadboard" Voltage Regulator		    				& 4.0						& 0\%				\\	
                    \hline
		\end{tabular}
	\end{table}

	%\subsubsection*{Special Problems or other reporting}
    %Predicting hours seems to be a problem as we are usually underestimating the time it takes us to complete tasks. I have been lucky so far as I have not had much strain from other sources, but in the future I will need to allocate time more efficiently.
     
	

%| =================================================================================================
%| Individual Activity Synopsis
%| Devin
%| =================================================================================================
\subsection{Devin Moore}

	\subsubsection*{Synopsis of past week's work}

	The beginning of the week consisted of researching and discussing possible features for the system. 
	Once they had been compiled we broke them up into groups to research and report the various
	aspects of each feature including why it is relevant, what has been done before, how it would be
	implemented, and resources needed. After we had finished, we began typesetting the document in \LaTeX\
	and refining our desired layout. There was a substantial amount of discussion on required features and 
    possible parts to order.


	\begin{table}[ht]
	\renewcommand{\arraystretch}{1.3}
		\caption{Devin Moore: Tasks Assigned - Last Week}
		
		\label{Summary of Devin Moore's activities: last week}
		
		\centering
		\begin{tabular}{p{5.5cm}|p{1cm}|p{1cm}}

		\hline
		\bfseries 	Task		 		                    & \bfseries Hours Worked	& \bfseries Status	\\
		\hline\hline
					Refine Problem Statement				& 2.0						& 80\%				\\
					Refined Design Contract					& 4.0						& 100\%				\\
					Selected and Ordered FPGA dev kit		& 3.0						& 70\%				\\
					Wrote Work Breakdown Structure			& 15.0						& 100\%				\\
					Started IMU I$^2$C Interface				& 4.0						& 15\%				\\
					Started Cell Phone App					& 2.0						& 10\%				\\
		\hline
		\end{tabular}
	\end{table}

	\begin{table}[ht]
	\renewcommand{\arraystretch}{1.3}
		\caption{Devin Moore: Tasks Assigned - Next Week}
		
		\label{Summary of Devin Moore's activites: this week}
		
		\centering
		\begin{tabular}{p{5.5cm}|p{1cm}|p{1cm}}

		\hline
		\bfseries 	Task		 		            & \bfseries Hours Worked	& \bfseries Status	\\
		\hline\hline
			    	Scheduling and Gantt Chart	    & 10.0						& 90\%				\\
                    IMU I$^2$C interface               & 25.0                      & 80\%              \\
                    \hline
		\end{tabular}
	\end{table}

	%\subsubsection*{Special Problems or other reporting}
	%Difficulty understanding what is actually due. The difference between the design document and the design
    %contract was not clear to me.
%| =================================================================================================
%| Individual Activity Synopsis
%| Ben
%| =================================================================================================
\subsection{Ben Smith}

	\subsubsection*{Synopsis of past week's work}
	I spent a good deal of time learning how use Mentor's Modelsim application and learning to write 
	a proper System Verilog test bench. Until now I have largely used Signaltap to validate design, 
	this was effective for small projects but is ineffective for large designs. A proper test bench 
	allows the use of automated debugging constructs like random generation and assertion which 
	ensure adherence to a much more rigorous specification and allow a more rapid verification-development 
	cycle. I averaged a 5 minute synthesis time average with the Cyclone over my summer internship. 
	This week’s compilations with Modelsim averaged under 5 seconds. The advantage of simulator 
	use is clear, more code, better code, less time.
	
	The analysis of the open cores module brought the group up to speed with I$^2$C. We have a clear
	understanding of how the I$^2$C protocol works. This understanding allows us to write a specification
	for what we need our module to do. We need a Master controller that will operate in single master
	mode at the 100kHz I$^2$C spec. We know both the accelerometer and gyroscope operate at this speed.

	Devin and I have begun to write the HDL for our implementation of a I$^2$C master controller. This
	week setup the skeleton of the State machine to control data transmission. The Input/Output port
	were defined in the top level module with support for the bidirectional SDA port.

	A different interface type for the heart rate strap was investigated. ANT+ is a open standard for
	personal area networks that already include a specification for a heart rate monitor. ANT+ is
	being adopted as the defacto standard for consumer biometric devices. Garmin and Polar both use
	this standard for their newer models of heart rate straps. A dual Bluetooth ANT+ module was ordered
	for experimentation. I already use a Garmin system and own a ANT+ heart rate strap and a speed/cadence
	module. Using Bluetooth to connect to both the heart rate and cellphone requires the bluetooth adapter
	to operate in master mode which will cause some implementation headaches. Using ANT+ allows the bluetooth
	adapter to operate in slave mode to connect to the cellphone and ANT+ for devices that need to communicate
	with the FPGA alone.

	\begin{table}[ht]
	\renewcommand{\arraystretch}{1.3}
		\caption{Ben Smith: Tasks Assigned - Last Week}
		
		\label{Summary of Ben Smith's activities: last week}
		
		\centering
		\begin{tabular}{p{5.5cm}|p{1cm}|p{1cm}}

		\hline
		\bfseries 		Task				        & \bfseries Hours Worked	& \bfseries Status	\\
		\hline\hline
					Refine Problem Statement		& 2.0						& 80\%				\\
                    Revising Design Contract        & 10.0                      & 80\%              \\
                    Work Breakdown Structure        & 10.0                      & 100\%            	\\
                    Verilog I$^2$C interface           & 15.0                      & 15\%              \\
		\hline
		\end{tabular}
	\end{table}

	\begin{table}[ht]
	\renewcommand{\arraystretch}{1.3}
		\caption{Ben Smith: Tasks Assigned - Next Week}
		
		\label{Summary of Ben Smith's activites: this week}
		
		\centering
		\begin{tabular}{p{5.5cm}|p{1cm}|p{1cm}}

		\hline
		\bfseries 	Task	            	 		& \bfseries Hours Worked	& \bfseries Status	\\
		\hline\hline
					Scheduling and Gantt Chart	    & 10.0						& 90\%				\\
                    IMU I$^2$C interface               & 25.0                      & 80\%              \\
                    ANT+ experimentation 		    & 10.0						& 10\%				\\
		\hline
		\end{tabular}
	\end{table}

	%\subsubsection*{Special Problems or other reporting}
    %I'm excited to start writing code.
\end{document}