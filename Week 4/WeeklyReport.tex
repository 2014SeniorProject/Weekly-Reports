%| Weekly report template for CSUS Senior Design
%|
%| language: LaTeX
%| Author: Ben Smith
%| 
%| This source has been tagged with the "<CHANGE" tag in areas
%| that require updating when making a new docuent
%|
%| This source will generate a PDF file complete with thumbnails navigation menu and metadata.
%| Much of the tex awesomeness comes from http://www.michaelshell.org/ praise be to him for creating the guide

\documentclass[12pt,compsoc]{IEEEtran}

%| Override compsoc class' Palatino font for body text, restores to Times New Roman
\renewcommand{\rmdefault}{ptm}\selectfont

%| IEEE Citation package
 \usepackage{cite}

%| American Mathematical Society package for fancy maths	
\usepackage[cmex10]{amsmath}
\interdisplaylinepenalty=2500				% Restores IEEE line spacing after amsmath

%| Better tables than LaTeX 2e
\usepackage{array}

%| Improved URL handling
\usepackage{url}

%| Enables PDF metadata, thumbnails, and navigation
\newcommand\MYhyperrefoptions{
	bookmarks=true,
	bookmarksnumbered=true,
	pdfpagemode={UseOutlines},
	plainpages=false,
	pdfpagelabels=true,
	colorlinks=true,
	linkcolor={black},
	citecolor={black},
	urlcolor={black},
	pdftitle={Senior Design: Weekly Progress Report},							%#CHANGE
	pdfsubject={Senior Design},													%#CHANGE
	pdfauthor={Michael Frith, David Larribas, Devin Moore, Benjamin Smith},		%#CHANGE
	pdfkeywords={Weekly Report}}												%#CHANGE

%| Calls hyper ref package 
\usepackage[\MYhyperrefoptions,pdftex]{hyperref}

% correct bad hyphenation here
\hyphenation{op-tical net-works semi-conduc-tor}

\begin{document}

%| =================================================================================================
%| Title Block
%| =================================================================================================
\title{Senior Design: Week Four Progress Report}
\author{Team 3: Micheal Frith, David Larribas, Devin Moore, Benjamin Smith}
\date{\today}
\maketitle

%| =================================================================================================
%| Header 
%| =================================================================================================
\markboth{Progress Report 4: \today}
{derp}

%| =================================================================================================
%| Weekly time report
%| =================================================================================================
\section{Group Time Synopsis}
	\subsection{Team efforts and overall status}
		{\bfseries Current Group Leader:} Ben Smith
		\IEEEPARstart{O}{nce} again the group spent the majority of their bandwidth on planning. We've 
		completed the Gantt chart and PERT critical path style graphs. Despite the time spent on the 
		PERT assignment the group has moved forward on the core project considerably this week.  Mike 
		ordered and received our motor and controller which will be used in the upcoming breadboard proof. 
		Devin has successfully communicated with the EEPROM I$^2$C modules on the Nano development board 
		through the modification of example code. I have begun to write the controller for the general 
		I$^2$C driver now that the byte level controller is complete. 

			Devin and I followed this split development path as a mitigation strategy to ensure we 
		could get data from the accelerometer for the breadboard proof. The general driver is a much 
		more complicated undertaking with a much higher chance of failure. Devin is going to continue 
		development on the general driver this week now that the example code has been modified to suit
		our needs.

			Our objective was to have the motor running and controlled by the accelerometer for the 
		breadboard proof. We are rapidly approaching this goal and we expect to be able to demonstrate 
		this capability within the next week.
	
	\begin{table}[ht]
		\renewcommand{\arraystretch}{1.3}
		\caption{Weekly Team Meetings}
		
		\label{Weekly Team Meetings}
		
		\centering
		\begin{tabular}{p{6.5cm}|p{1cm}}
		\hline
		{\bfseries 	Topic} 																	&{\bfseries Hours}		\\
		\hline\hline	
					{\bfseries Thursday:} PERT Planning and Section assignment				& 2	 					\\
					{\bfseries Saturday:} [Online] PERT first draft							& 3						\\
					{\bfseries Monday:} Weekly planning, Finalize WBS and weekly report		& 5	 					\\

		\hline
		\end{tabular}
	\end{table}
				
	\subsection{PERT assignment review and section assignment}
		The team discussed their individual sections and created a template to be filled out that describes each
		of the individual tasks. This task list formed the basis of our PERT network chart and the Gantt chart.

	\subsection{PERT and Gantt Chart First Draft}
		Team met online to synchronize how we described our individual tasks. The task list allowed us to generate
		the PERT and Gantt charts quickly.

		\subsection{Weekly planning, Finalize WBS and weekly report}
		Gantt chart was finalized and put on Bulletin board for review. Individual tasks reviewed and a plan for 
		breadboard proof preparation was completed. Mike and David will work together to prepare the motor and
		controller. Ben and Devin will prepare the accelerometer based feedback algorithm and I$^2$C communication
		module.

	\begin{table}[ht]
		\renewcommand{\arraystretch}{1.3}
		\caption{Team Hour Summary}
		
		\label{Team Hour Summary}
		
		\centering
		\begin{tabular}{p{5cm}|p{1.5cm}}
		\hline
		{\bfseries 	Topic} 								&{\bfseries Hours}		\\
		\hline\hline
					{\bfseries Michael Frith}  			& 19.0					\\
					{\bfseries David Larribas} 			& 15.0					\\
					{\bfseries Devin Moore}				& 25.0					\\
					{\bfseries Ben Smith}  				& 28.0					\\		
					\hline
		\end{tabular}
	\end{table}

%| =================================================================================================
%| Individual Activity Synopsis
%| Mike
%| =================================================================================================
\section{Individual Activity Synopsis}
	\subsection{Michael Frith}

	\subsubsection*{Synopsis of past week's work}
        This past week I finalized some component selection and was able to talk a vendor into selling 
        us a used motor at a discount. I also finalized the the ESC selection and ordered that unit 
        as well. We went over and revised our punch list and design contract after input from Professor 
        Tatro. I spent time working on our task list and scheduling as well.

    \begin{table}[ht]
		\renewcommand{\arraystretch}{1.3}
		\caption{Micheal Frith: Tasks Assigned - Last Week}
		
		\label{Summary of Micheal Frith's activities: last week}
		
		\centering
		\begin{tabular}{p{5.5cm}|p{1cm}|p{1cm}}
		\hline
		\bfseries 	Task		 								& \bfseries Hours Worked	& \bfseries Status	\\
		\hline\hline
					Motor selected and ordered 					& 4.0						& 100\%				\\
					ESC Motor Controller selected \& ordered   	& 3.0		    			& 100\% 			\\
					Feature list presentation 					& 2.0						& 100\%				\\
					Design Contract \& Punch list Revision		& 3.0						& 100\%				\\
					List of tasks and descriptions				& 5.0						& 80\%				\\
					Gantt and Pert Chart						& 4.0						& 80\%				\\
					Weekly Report								& 4.0						& 100\%				\\
		\end{tabular}
	\end{table}

	\begin{table}[ht]
	\renewcommand{\arraystretch}{1.3}
		\caption{Micheal Frith: Tasks Assigned - This Week}
		
		\label{Summary of Micheal Frith's activities: this week}
		
		\centering
		\begin{tabular}{p{5.5cm}|p{1cm}|p{1cm}}
		\hline
		\bfseries 	Task		 							& \bfseries Hours Worked	& \bfseries Status	\\
		\hline\hline
					Motor installation and testing 			& 4.0						& 100\%				\\
					ESC installation and testing			& 4.0						& 100\%				\\
					Breadboard proof preparation			& 6.0						& 100\%				\\
					Weekly Report							& 1.0						& 100\%				\\
		\hline
		\end{tabular}
	\end{table}

%| =================================================================================================
%| Individual Activity Synopsis
%| David
%| =================================================================================================
\subsection{David Larribas}

	\subsubsection*{Synopsis of past week's work}

	This week we refined our feature set after we were corrected during our presentation.  After spending 
	a day refining the list, I spent much of my time building the task list and the gantt chart.  I spent 
	a few free hours comparing buck converters and linear conversion for the voltage regulator. It was a 
	quick decision to go with the buck so I searched through half a dozen or so websites learning about 
	buck circuits.

	\begin{table}[ht]
	\renewcommand{\arraystretch}{1.3}
		\caption{David Larribas: Tasks Assigned - Last Week}
		
		\label{Summary of David Larribas' activities: last week}
		
		\centering
		\begin{tabular}{p{5.5cm}|p{1cm}|p{1cm}}
		\hline
		\bfseries 	Task		 	                        				& \bfseries Hours Worked	& \bfseries Status	\\
		\hline\hline
					Refine Feature Set										&2.0 						&100\%				\\
					Develop Gantt Chart										&8.0 						&80\%				\\
					Component Research and Selection for Lighting			&0.0 						&80\%				\\
					Component Research and Selection for Voltage Regulator	&2.0 						&35\%				\\
					“Breadboard” Voltage Regulator							&0.0 						&0\%				\\
		\hline
		\end{tabular}
	\end{table}

	\begin{table}[ht]
	\renewcommand{\arraystretch}{1.3}
		\caption{David Larribas: Tasks Assigned - Next Week}
		
		\label{Summary of David Larribas' activites: this week}
		
		\centering
		\begin{tabular}{p{5.5cm}|p{1cm}|p{1cm}}
		\hline

		\bfseries 	Task		 	                         			    & \bfseries Hours Worked	& \bfseries Status	\\
		\hline\hline
					Build Motor Mount										& 4.0 						& 5\%				\\
					Component Research and Selection for Lighting			& 2.0 						& 50\%				\\
					Component Research and Selection for Voltage Regulator	& 8.0 						& 25\%				\\
					“Breadboard” Voltage Regulator							& 4.0 						& 0\%				\\
					Individual Weekly Report								& 1.0 						& 100\% 			\\
                    \hline
		\end{tabular}
	\end{table}   

%| =================================================================================================
%| Individual Activity Synopsis
%| Devin
%| =================================================================================================
\subsection{Devin Moore}

	\subsubsection*{Synopsis of past week's work}

	This week I began developing a simple I$^2$C Verilog module to test to the protocol. I was able to 
	create a master module to read and write to the EEPROM on the De0-Nano for testing. I then set  
	up our Gantt chart and a document to keep track of the individual task descriptions. I wrote 
	descriptions for all of my tasks and added them to the Gantt chart. Once all of the task descriptions
	were finished I typeset the the LaTeX document. I then helped create the PERT Chart.


	\begin{table}[ht]
	\renewcommand{\arraystretch}{1.3}
		\caption{Devin Moore: Tasks Assigned - Last Week}
		
		\label{Summary of Devin Moore's activities: last week}
		
		\centering
		\begin{tabular}{p{5.5cm}|p{1cm}|p{1cm}}

		\hline
		\bfseries 	Task		 		                    & \bfseries Hours Worked	& \bfseries Status	\\
		\hline\hline
					Feature list presentation 				& 2.0 						& 100\%				\\
					Design Contract Punchlist Revision		& 3.0 						& 100\%				\\
					List of Tasks and Descriptions			&10 						& 100\%				\\
					PERT and Gantt Chart					&9.0 						& 90\%				\\
					Weekly Report							&1.0 						& 100\%				\\
					Verilog I2C Developement				&8.0						& 80\%				\\
		\hline\
		\end{tabular}
	\end{table}

	\begin{table}[ht]
	\renewcommand{\arraystretch}{1.3}
		\caption{Devin Moore: Tasks Assigned - Next Week}
		
		\label{Summary of Devin Moore's activites: this week}
		
		\centering
		\begin{tabular}{p{5.5cm}|p{1cm}|p{1cm}}

		\hline
		\bfseries 	Task		 		            & \bfseries Hours Worked	& \bfseries Status	\\
		\hline\hline
                    IMU I$^2$C interface            & 15.0                      & 80\%              \\
                    Preparing for breadboard proof  & 10.0                      & 80\%              \\
                    Individual Weekly Report		& 1.0 						& 100\% 			\\
                    \hline
		\end{tabular}
	\end{table}

%| =================================================================================================
%| Individual Activity Synopsis
%| Ben
%| =================================================================================================
\subsection{Ben Smith}

	\subsubsection*{Synopsis of past week's work}
	Using last week’s development of Modelsim skill I began the development of our I$^2$C implementation and
	it's testing framework. The “byte level” I$^2$C controller has been completed and tested in a simulation environment. This module 
    directly manipulates the I$^2$C data and clock lines. The controller implements what we have begun 
	to call “phase one” of our controller - single-master with no SCL stretching or arbitration. The more 
	advanced features of the I$^2$C spec are to be implemented after the breadboard proof. 

	The low level controller is accompanied by another System Verilog test bench that verifies its operation 
	with a simulated slave controller. Next week the top level controller module will be developed which will 
	enable allow the slave module to make arbitrary writes and reads from a slave register. Between the three
	modules about 350 lines of code were produced last week. We should be able have this module ready for use
	by the breadboard proof. If not, we already have a working module from adapted Terasic example code 
	that Devin produced as a risk mitigation plan. 

	\begin{table}[ht]
	\renewcommand{\arraystretch}{1.3}
		\caption{Ben Smith: Tasks Assigned - Last Week}
		
		\label{Summary of Ben Smith's activities: last week}
		
		\centering
		\begin{tabular}{p{5.5cm}|p{1cm}|p{1cm}}

		\hline
		\bfseries 		Task				        & \bfseries Hours Worked	& \bfseries Status	\\
		\hline\hline
                    Revising Design Contract        & 2.0                       & 80\%              \\
                    Verilog I$^2$C interface        & 20.0                      & 40\%              \\
                    Typeset Weekly report 			& 1.0						& 100\% 			\\
                    Individual Weekly Report		& 1.0 						& 100\% 			\\
                    Group Leader Weekly Report 		& 1.0						& 100\% 			\\
		\hline
		\end{tabular}
	\end{table}

	\begin{table}[ht]
	\renewcommand{\arraystretch}{1.3}
		\caption{Ben Smith: Tasks Assigned - Next Week}
		
		\label{Summary of Ben Smith's activites: this week}
		
		\centering
		\begin{tabular}{p{5.5cm}|p{1cm}|p{1cm}}

		\hline
		\bfseries 	Task	            	 		& \bfseries Hours Worked	& \bfseries Status	\\
		\hline\hline
                    IMU I$^2$C interface            & 20.0                      & 80\%              \\
                    Preparing for breadboard proof  & 10.0                      & 100\%              \\
                    Individual Weekly Report		& 1.0 						& 100\% 			\\
                    Group Leader Weekly Report 		& 1.0						& 100\% 			\\
		\hline
		\end{tabular}
	\end{table}
\end{document}